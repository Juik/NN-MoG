\documentclass[12pt]{article}

\usepackage {fullpage}
%\usepackage {palatino}
\usepackage {epsfig}
\usepackage{amsmath}
\usepackage{amsfonts}
\textwidth 6.5in
\topmargin 0in \headheight 0in
%\headsep 0.5in
\textheight 8.7in
\oddsidemargin 0in
\evensidemargin 0in
\parskip 0.1in
\parindent 0in

\pagestyle{empty}

\def\beqa#1\eeqa{\begin{eqnarray}#1\end{eqnarray}}
\newcommand{\bmu}{{\boldsymbol{\mu}}}
\newcommand{\bSigma}{{\boldsymbol{\Sigma}}}
\newcommand{\bx}{{\bf{x}}}


\newcommand{\items}{\vspace{-.25in}\begin{itemize}\setlength{\itemsep}{1mm}}
\newcommand{\degree}{{\raisebox{.6ex}[0ex][0ex]{\small $\circ$}}}
\newcommand{\cut}[1]{}
\newcommand{\eg}{{\it eg \/}}
\newcommand{\ie}{{\it ie \/}}
\newcommand{\bighead}[1]{\begin{slide}\begin{center}{\bf \large
\underline{#1}}\end{center}}
\newcommand{\numbs}{\vspace{-.25in}\begin{enumerate}\setlength{\itemsep}{1mm}}
\newcommand{\itemss}{\vspace{-.05in}\begin{itemize}\setlength{\itemsep}{1mm}}

\newcommand{\indentit}{\mbox{  }\hspace{.1in}}
\newcommand{\vbon}{\begin{verbatim}}
\newcommand{\vboff}{\end{verbatim}}
\renewcommand{\vec}[1]{\boldsymbol{#1}}
\newcommand{\yy}{\mathbf{y}}
\newcommand{\xx}{\mathbf{x}}
\newcommand{\zz}{\mathbf{z}}
\newcommand{\ttt}{\mathbf{t}}
\newcommand{\dd}[2]{\frac{\partial #1}{\partial #2}}
\newcommand{\pderi}[2]{\frac{\partial #1}{\partial #2}}
\newcommand{\deri}[2]{\frac{{\rm d} #1}{{\rm d} #2}}
\DeclareMathOperator{\E}{\mathbb{E}}
\DeclareMathOperator{\cov}{cov}
\DeclareMathOperator{\diag}{diag}

\begin{document}

\begin{center}
{\bf CSC 411/2515}\\
{\bf Machine Learning and Data Mining}\\
{\large \bf Assignment 2}\\
{\bf Out:  Oct. 28}\\
{\bf Due:  Nov 11 [noon]}
\end{center}

\section*{Overview}

In this assignment, you will build a decision tree, and experiment with a
neural network and a naive Bayes model. Some code that implements a neural
network with one hidden layer, and the naive Bayes model will be provided for you (both MATLAB and Python).

You will be working with the following dataset:

{\bf Digits}: The file {\tt digits.mat} contains 6 sets of $16 \times 16$
greyscale images in vector format (the pixel intensities are between 0 and 1 and
were read into the vectors in a raster-scan manner). The images contain
centered, handwritten 2's and 3's, scanned from postal envelopes. {\tt train2}
and {\tt train3} contain examples of 2's and 3's respectively to be used for
training. There are 300 examples of each digit, stored as $256 \times 300$
matrices. Note that each data vector is a column of the matrix. {\tt valid2} and {\tt valid3} contain data to be used for validation
(100 examples of each digit) and {\tt test2} and {\tt test3} contain test data
to be used for final evaluation {\bf only} (200 examples of each digit).

\section{EM for Mixture of Gaussians (15 pts) }

Let us consider a Gaussian mixture model:
\beqa
 p(\bx) = \sum_{k=1}^K \pi_k {\cal{N}}(\bx | \bmu_k, \bSigma_k)
\eeqa

Consider a special case of a Gaussian mixture model in which the covariance matrices
$\bSigma_k$ of the components are all constrained to have a common value $\bSigma$. 
In other words $\bSigma_k = \bSigma$, for all $k$. 
Derive
the EM equations for maximizing the likelihood function under such a model.


\section{Neural Networks (40 points)}

Code for training a neural network with one hidden layer of logistic units,
logistic output units and a cross entropy error function is included. The main components are:

MATLAB
\begin{itemize}
\item {\tt init\_nn.m}: initializes the weights and loads the training, validation and test data.
\item {\tt train\_nn.m}: runs {\tt num\_epochs} of backprop learning.
\item {\tt test\_nn.m}: Evaluates the network on the test set.
\end{itemize}

Python
\begin{itemize}
\item {\tt nn.py} : Methods to perform initialization, backprop learning and
  testing.
\end{itemize}

\subsection{Basic generalization [8 points]}

Train a neural network with 10 hidden units. You should
first use {\tt init\_nn} to initialize the net, and then execute {\tt train\_nn}
repeatedly (more than 5 times). Note that {\tt train\_nn} runs 100 epochs each
time and will output the statistics and plot the error curves. Alternatively, if
you wish to use Python, set the appropriate number of epochs in {\tt nn.py} and
run it. Examine the statistics and plots of training error and validation error (generalization).
How does the network's performance differ on the training set versus the validation set
during learning? Show a plot of error curves (training and validation) to support your argument.

\subsection{Classification error [8 points]}

You should implement an alternative performance measure to the cross entropy, the mean
classification error. You can
consider the output correct if the correct label is given a higher probability
than the incorrect label.  You should then count up the total number of
examples that are classified incorrectly according to this criterion for
training and validation respectively, and maintain this statistic at the end of
each epoch. Plot the classification error vs. number of epochs, for both
training and validation.

\subsection{Learning rate [8 points]}

Try different values of the learning rate $\epsilon$ (``eps'') defined in {\tt
init\_nn.m} (and in {\tt nn.py}). You should reduce it to $.01$, and increase it to $0.2$ and
$0.5$. What happens to the convergence properties of the algorithm (looking at
both cross entropy and \%Correct)? Try momentum of $\{0.0, 0.5, 0.9\}$. How does
momentum affect convergence rate ? How would you choose the best value of these
parameters?

\subsection{Number of hidden units [8 points]}

Set the learning rate $\epsilon$ to $.02$, momentum to $0.5$ and try different numbers of
hidden units on this problem (you might also need to adjust
\texttt{num\_epochs} accordingly in \texttt{init\_nn.m}). You should use two values $\{2, 5\}$, which are
smaller than the original and two others $\{30, 100\}$, which are larger.
Describe the effect of this modification on the convergence properties, and the
generalization of the network.

\subsection{Compare $k$-NN and Neural Networks (8 points)}

Try $k$-NN on this digit classification task using the code provided in the
first assignment, and compare the results with those you got using neural
networks and naive Bayes. Briefly comment on the differences between
these classifiers.



\section{Mixtures of Gaussians (45 points)}

\subsection{Code}

The Matlab file {\tt mogEM.m} implements the EM algorithm for the MoG model.\\
The file {\tt mogLogProb.m} computes the log-probability of data under a MoG model.\\
The file {\tt kmeans.m} contains the k-means algorithm.\\
The file {\tt distmat.m} contains a function that efficiently computes pairwise distances between
sets of vectors.  It is used in the implementation of k-means. \\

Similarly, {\tt mogEM.py} implements methods related to training MoG models.\\
The file {\tt kmeans.py} implements k-means.\\

As always, read and understand code before using it.

\subsection{Training (15 points)}

\label{sec:training}



The Matlab variables {\tt train2} and {\tt train3} each contain 300 training
examples of handwritten 2's and 3's, respectively.
Take a look at some of them to make sure you have transferred the data properly.
In Matlab, plot the digits as images using {\tt imagesc(reshape(vector,16,16))}, which converts a
256-vector to an 16x16 image.  You may also need to use {\tt colormap(gray)} to obtain grayscale image.
Look at {\tt kmeans.py} to see an example of how to do this in Python.


For each training set separately, train a mixture-of-Gaussians using the code in {\tt
mogEM.m}. Let the number of clusters in the Gaussian mixture be 2, and the
minimum variance be 0.01. You will also need to experiment with the
parameter settings, e.g. {\tt randConst}, in that program to get sensible clustering results.
And you'll need to execute {\tt mogEM} a few times for each digit, and see the local
optima the EM algorithm finds. Choose a good model for each digit from your
results.



For each model, show both the mean vector(s) and variance vector(s) as
images, and show the mixing proportions for the clusters within each model.
Finally, provide $\log P(Training Data)$ for each model.


\subsection{Initializing a mixture of Gaussians with k-means (10 points)}

Training a MoG model with many components tends to be slow.  People have found that
initializing the means of the mixture components by running a few iterations of k-means
tends to speed up convergence. You will experiment with this method of initialization.
You should do the following.


\begin{itemize}
\item Read and understand {\tt kmeans.m} and {\tt distmat.m} (Alternatively,
{\tt kmeans.py}).

\item Change the initialization of the means in {\tt mogEM.m} (or {\tt mogEm.py}) to use the k-means algorithm.
As a result of the change the model should run k-means on the training data and use the returned means as the
starting values for {\tt mu}.  Use 5 iterations of k-means.

\item Train a MoG model with 20 components on all 600 training vectors (both 2's and 3's) using
both the original initialization and the one based on k-means.
Comment on the speed of convergence as well as the final log-prob resulting from
the two initialization methods.



\end{itemize}



\subsection{Classification using MoGs (20 points)}

Now we will investigate using the trained mixture models for
classification. The goal is to decide which digit class $d$ a new
input image $\xx$ belongs to.  We'll assign $d=1$ to the 2's and
$d=2$ to the 3's.

For each mixture model, after training, the likelihoods $P(\xx|d)$ for
each class can be computed for an image $\xx$ by consulting
the model trained on examples from that class; probabilistic
inference can be used to compute $P(d|\xx)$, and the most probable
digit class can be chosen to classify the image.

Write a program that computes $P(d=1|\xx)$ and $P(d=2|\xx)$ based
on the outputs of the two trained models. You can use {\tt
mogLogProb.m} (or the method {\tt mogLogProb} in {\tt mogEm.py}) to compute the
log probability of examples under any model.

You will compare models trained with the same number of mixture components. You
have trained 2's and 3's models with 2 components.  Also train models with more
components: 5, 15 and 25.  For each number, use your program to classify
the validation and test examples.

For each of the validation and test examples, compute $P(d|\xx)$ and classify the
example. Plot the results. The plot should have 3 curves of
classification error rates versus number of mixture components
(averages are taken over the two classes):


\begin{itemize}
\item The average classification error rate, on the training set
\item The average classification error rate, on the validation set
\item The average classification error rate, on the test set
\end{itemize}



%\vspace{.1in}


Provide answers to these questions:

\vspace{.2in}

\numbs


\item You should find that the error rates on the training sets generally
decrease as the number of clusters increases. Explain why.

\item Examine the error rate curve for the test set and discuss its properties. Explain the trends that you observe.





\item If you wanted to choose a particular
model from your experiments as the best, how would you choose it?
If your aim is to achieve the lowest error
rate possible on the new images your system will receive, which model
(number of clusters) would you select? Why?



\end{enumerate}




\subsection{Bonus Question: Mixture of Gaussians vs Neural Network (10 points)}


Choose the best mixture of Gaussian classifier you have got so far according to
your answer to question 3 in the section above. Compare this mixture of Gaussian
classifier with the neural network. For this comparison, set the number of
hidden units equal to the number of mixture components in the mixture model
(digit 2 and 3 combined). Visualize the input to hidden weights as images to see
what your network has learned. You can use the same trick as mentioned in
section 2 to visualize these vectors as images.



You can visualize how the input to hidden weights change during training.
Discuss the classification performance of the two models and compare the hidden
unit weights in the neural network with the mixture components in the mixture
model.





\section{Write up}

Hand in answers to all the questions in the parts above. The goal of your write-up is to document
the experiments you have done and your main findings. So be sure to explain the results.
The answers to your questions should be in pdf form and turned in along with your code. Package
your code and a copy of the write-up pdf document into a zip or tar.gz file called
A1-*your-student-id*.[zip|tar.gz]. Only include functions and scripts that you 
modified. Submit this file on MarkUs. Do not turn in a hard copy of the write-up.


%Hand in answers to all the questions in the parts above.  The goal of your
%write-up is to document the experiments you've done and your main findings.  So
%be sure to explain the results.
%It is up to you whether you would like to do the write-up of Part 1 (Decision
%Trees) by hand or on the computer.  In either case you should turn in a
%hardcopy of your answer to Part I.  The answers to the rest of the questions
%should be in pdf form, and turned in along with your code.

%Package your code and a copy of the write-up \texttt{pdf} document using
%\texttt{zip} or \texttt{tar.gz} in a file called 
%\texttt{CSC411-A2-*your-student-id*.[zip|tar.gz]}. Only include functions and
%scripts that you modified. 
%Submit this on MarkUs by October 17, 2014, noon. 
%Turn in a hard copy of the write-up of Part 1 before class on October 17
%as well.

\end{document}

